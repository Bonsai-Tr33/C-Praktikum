%Diese LaTeX-Vorlage f�r Praktikumsprotokolle in Form einer Ver�ffentlichung f�r das 
%F-Praktikum wurde erstellt von Andreas Nuber EP II, Uni W�rzburg.
%Es wurde das Koma-Skript verwendet und sollte somit installiert sein
%desweiteren sollten alle packages installiert sein, die mit \usepackage{} 
%eingebunden werden.
%Zum Testen dieser Vorlage wurde MikTex verwendet sowie TeXnicCenter als Editor
%beim Compilieren waren 0 Fehler, 1 Warnung und 3 zu volle/leere Boxen. Das ist ok :)

%F�r das Erstellen einfach den sinnfreien Text, der zum ausf�llen genommen wurde 
%ersetzen. Was sonst noch ver�ndert werden sollte steht in den Kommentaren!


\documentclass[a4paper,10pt,twocolumn]{scrartcl} %Koma-Skript-�quivalent zu "article"

\usepackage{german}            %macht deutsche �berschriften
\usepackage{amsmath}           %macht
\usepackage{amsfonts}          %       Mathe
\usepackage{amssymb}           %              m�chtiger
\usepackage{graphicx}          %erlaubt Graphiken einzubinden (.eps f�r dvi und ps sowie .jpg f�r pdf)
\usepackage[T1]{fontenc}       %Zeichenbelegung der verwendeten Schrift
\usepackage{ae}                %macht sch�neres �
\usepackage{typearea}	         %erm�glicht �nderung des Seitenspiegels
\usepackage{scrlayer-scrpage}          %erm�glicht �nderung der Kopf-/Fu�zeile
\usepackage{lastpage}         %l�sst auf die Seienanzahl zugreifen
\usepackage[margin=10pt,font=small,labelfont=bf]{caption} %macht die Bildbeschriftungen richtig
\usepackage{xcolor}			  % macht Farbe
\usepackage{subfig}
\usepackage{hyperref}
\usepackage{float}
\usepackage{comment}

\renewcommand{\figurename}{Abb.}

\pagestyle{scrheadings}        %sagt Koma-Skript, dass selbstdefiniers Kopfzeilen verwendet werden
\typearea{16}                  %stellt Seitenspiegel ein
\columnsep25pt								 %definiert Breite zwischen den zwei Spalten von \twocolumns

\renewcommand{\pnumfont}{%     %�ndert die Schriftart der Seitennummerierung
\normalfont\rmfamily\slshape}  %�ndert die Schriftart der Seitennummerierung 



\begin{document}
\cfoot{\thepage /\pageref{LastPage}}     %macht die Seitennumerierung der Fom 2/5 (ausser auf der Titelseite)

\twocolumn[{\csname @twocolumnfalse\endcsname                %erlaubt "Abstrakt" �ber beide Spalten
\titlehead{                                                  %Kopfzeile
	\begin{tabular*}{\textwidth}[]{@{\extracolsep{\fill}}lr}   %Kopfzeile
	Betreuer: Prof. Dr. V. Hinkov & \today\\                          %Kopfzeile      hier den Betreuer eintragen!!!
	\end{tabular*}                                             %Kopfzeile
	}
\title{Messung der Ausbreitungsgeschwindigkeit von elektromagnetischen Wellen auf Kabeln}  %Titel der Versuchs
\author{Hannes Winkler \and Moritz Langer}                     %Namen der Studenten
\date{}			                                               %ben�tigt um automatisches Datum auszuschalten
\maketitle                                                      %erzeugt Titelseite
\vspace{-8ex}                                                   %verringert Abstand zur �berschrift
\begin{abstract}                                                %Beginn des Abstracts
\vspace{1em}
\noindent Versuchsdurchführung: 06. November 2025\\% Datum ändern!
\noindent Protokollabgabe: 20. November 2025% Datum ändern!
\vspace{1em}

	\noindent An einem Koaxialkabel des Typs RG213/U wird durch Reflexionsunterdrückung der Wellenwiderstand $Z = (50,0 \pm 1,0) \, \Omega$ bestimmt. Das Reflexionsverhalten wird durch benutzen
	verschiedener Abschlusswiderstände untersucht. An selbigem Kabel und an einem Verzögerungskabel wird durch Laufzeitmessung die Ausbreitungsgeschwindigkeit
	von Signalen gemessen und an dem RG213/U Koaxialkabel auch mittels stehender Wellen. Bei der Methode der stehenden Wellen lässt sich die Dispersion der 
	Ausbreitungsgeschwindigkeit und die Dämpfung im Kabel beobachten. Des weiteren wurde ebenfalls durch Laufzeitmessung die Lichtgeschwindigkeit bestimmt welche
	ein Wert von $c = (293 \pm 16)\cdot 10^6 \, \frac{m}{s}$ ergab.
\end{abstract}
}]

\section{Einleitung}

	Das Koaxialkabel hat sich aufgrund seiner besonderen elektromagnetischen Eigenschaften in der Informationsübertragung als sinnvolles Medium etabliert.
	Durch den Aufbau des Kabels mit Innenleiter, Dielektrikum und Außenleiter wird das elektromagnetische Feld in Ausbreitungsrichtung verringert bzw. komplett
	unterdrückt, was zu einer verringerten Dämpfung, Dispersion und auch Reflexion führt, was eine stabile und störungsfreie Signalübertragung ermöglicht.\\
	Diese Arbeit beschäftigt sich damit, diese Eigenschaften von Koaxialkabeln genauer zu beleuchten, deren Hintergrund zu verstehen und auch den Umgang mit Koaxialkabeln
	zu erlernen.\\
	Des weiteren wird eine zentrale Naturkonstante, die Lichtgeschwindigkeit, gemessen und diskutiert.
\section{Theorie}

	\subsection{Wellenausbreitung auf Kabeln}

		\subsubsection{Wellenwiderstand}

			Bei der Untersuchung von Kabeln gibt es zwei charakteristische Größen:
			Die relative Induktivität   $L^{\ast} = \frac{\text{Induktivität}}{\text{Längeneinheit}}$
			und die relative Kapazität $C^{\ast} = \frac{\text{Kapazität}}{\text{Längeneinheit}}$. Über
			deren Zusammenhang mit der Spannung 

			\begin{displaymath}
				U_{\pm}(x,t) = \sqrt{\frac{L^{\ast}}{C^{\ast}}} \cdot I_{\pm}(x,t) = Z \cdot I_{\pm}(x,t)
			\end{displaymath}
	
			\noindent lässt sich analog zu dem Ohmschen Gesetz der Wellenwiderstand $Z$ des Kabels definieren.
			Wichtig zu beachten ist, dass es sich nicht um einen ohmschen Widerstand mit Verlusten
			handelt, sondern um einen Wellenwiderstand welcher das Verhältnis zwischen
			Spannung und Strom in dem Kabel angibt.
		
		\subsubsection{Reflexion} \label{Reflexion}

			Sobald das Signal das Kabelende erreicht wird es reflektiert. Die Art bzw. das Maß
			der Reflexion hängt hierbei von dem Widerstand $R_v$ ab, welcher sich am Ende befindet.
			Dieses Maß nennt sich der Reflexionskoeffizient $\rho$ welcher definiert ist über 
			das Verhältnis zwischen der Amplitude der auslaufenden und der Amplitude der einlaufenden 
			Welle und kann mithilfe des Ohmschen Gesetzes weiter vereinfacht werden als:

			\begin{displaymath}
				\rho = \frac{U_{-}(x,t)}{U_{+}(x,t)} = \frac{R_v - Z}{R_v + Z}
			\end{displaymath}

			\noindent Anhand dieser Formel lassen sich drei Spezialfälle des Reflexionskoeffizient erkennen:

			\begin{enumerate}
				\item Kurzschluss der Leitung: \\
						Das reflektierte Signal entspricht dem einlaufenden Signal mit entgegengesetztem
						Vorzeichen. Die Wellen löschen sich am Kabelende aus\\
						Es gilt: $R_v = 0 \Omega \Rightarrow \rho = -1 $
				\item Offenes Ende der Leitung: \\
						Das reflektiert Signal ist gleich dem einlaufenden Signal und diese 
						addieren sich. \\
						Es gilt: $R_v = \infty \Rightarrow \rho = 1 $
				\item Reflexionsfreier Abschluss:
						Durch richtige Wahl des Abschlusswiderstandes bleibt die Reflexion aus und 
						es gibt nur eine einlaufende Welle. \\
						Es gild $R_v = Z \Rightarrow \rho = 0 $
			\end{enumerate}

		\subsubsection{Mathematische Beschreibung einer Welle} \label{MathWell}
			Wird durch ein Koaxialkabel eine sinusförmige Welle gesendet, kann diese am Kabelende reflektiert werden (siehe \ref{Reflexion}).
			Wenn diese Wellen mit gleicher Frequenz und Amplitude sich überlagern entsteht eine stehende Welle, 
			welche mathematisch als die Addition zweier gegenläufigen Wellen beschrieben wird:
			\begin{equation*}
			\begin{split}
				y(x,t) &= A \sin(kx - \omega t) + A \sin(kx + \omega t) \\
						&= 2A \sin(kx)\cos(\omega t)
			\end{split}
			\end{equation*}
			Die resultierende Welle hat (im Idealfall) Orte, an denen die Amplitude konstant 0 ist (sog. Knoten).
			Im Fall eines Koaxialkabels kommt es jedoch zu Dämpfung, weshalb die Knoten endliche Werte ungleich Null annehmen.
			In Abschnitt \ref{sec:Ausbreitungsgeschwindigkeit} wird mithilfe eines sinusförmigen Signals eine stehende Welle im Koaxialkabel erzeugt.
			Das Ende wird je nach Resonanzbedingung offen gelassen oder kurzgeschlossen.\\
			Bei offenem Kabel gilt:
			\begin{equation*}
				L = \frac{\lambda}{4}, \frac{3 \lambda}{4}, \frac{5 \lambda}{4}, \ldots
			\label{WellenlängeOffen}
			\end{equation*}
			Bei kurzgeschlossenem Kabel gilt:
			\begin{equation*}
				L = \frac{\lambda}{2}, \lambda, \frac{3 \lambda}{2}, \ldots
			\label{WellenlängeGeschlossen}
			\end{equation*}
			Die Ausbreitungsgeschwindigkeit der Welle ergibt sich dann (unabhängig von der Resonanzbedingung) volgendermaßen:
			\begin{equation*}
				v = \lambda \cdot f
			\label{Ausbreitungsgeschwindigkeit}
			\end{equation*}
			Durch lösen der eindimensionalen Wellengleichung und durch einsetzen von $L^{*}$ und $C^{*}$ ergibt sich zudem folgender Zusammenhang:
			\begin{equation*}
				v = \frac{1}{L^{\ast} \, C^{\ast}} = \frac{c}{\sqrt{\mu_r \epsilon_r}}
			\label{Ausbreitungsgeschwindigkeit2}
			\end{equation*}
			Hierbei ist c die Lichtgeschwindigkeit, $\mu_r$ die Permeabilität und $\epsilon_r$ die Dielektrizitätskonstante.\\
			Im Versuch werden die Frequenzen so gewählt, sodass ein Amplitudenminimum am Kabelanfang beobachtet werden kann. Aus der Kabellänge lässt sich dann die Wellenlänge
			und somit auch die Ausbreitungsgeschwindigkeit bestimmen.

\section{Experiment und Auswertung}

	\subsection{Signalübertragung auf Koaxialkabel bei verschiedenartiger Anpassung}

		\subsubsection{Beobachtung des Signalverlaufs}

		Zu Beginn wurde der Funktionsgenerator an das Oszilloskop angeschlossen, um den Verlauf des Graphen zu beobachten.
		Dazu wurde eine Peak-to-Peak-Spannung von $U_{pp} = 10 \, V$ eingestellt und eine Amplitude von $U_{pp} = 20 \, V$ am Oszilloskop beobachtet. 
		Da das Oszilloskop einen vergleichsweise hohen Widerstand hat ($R_{1}\rightarrow\infty$), entspricht das anschließen des Kabels einem offenen Ende.
		Das reflektierte Signal ist dann gleich dem einlaufenden und diese addieren sich wie in \ref{MathWell} erklärt. Um dieses Problem zu beheben, wird ein Widerstand
		$R_2 = 50 \, \Omega$ parallel dazu geschaltet. Dadurch ergibt sich nach \\ 
		$\frac{1}{R_{ges}} = \frac{1}{R_{1}} + \frac{1}{R-2}$ \\
		ein Widerstand von $R_{ges} = 50 \, \Omega$ am Ende des Koaxialkabels. Dieser Widerstand entspricht genau der Impedanz $Z = 50 \, \Omega$ des Kabels.
		Dadurch ergibt sich ein reflexionsfreier Abschluss und am Oszilloskop ist eine Amplitude von $U_{pp} = 10 \, V$ abzulesen.\\
		Wenn nicht explizit erwähnt, wurde der Widerstand in jedem der folgenden Versuchsteile eingebaut.
		
		\subsubsection{Messung des Wellenwiderstandes eines Koaxialkabels}

		Zunächst wurde am Funktionsgenerator eine Frequenz $f = 200 \, kHz$, eine Impulsbreite von $t = 20 \, ns$ und eine Spannung von $U_{pp} = 10 \, V$ eingestellt. Mittels T-Stecker
		wird das $50 \, m$ lange Koaxialkabel sowohl mit dem Funktionsgenerator als auch mit dem Oszilloskop verbunden. 
		Des weiteren schließt man an das generatorferne Ende des Kabels
		ein Kohleschichtpotentiometer. \\
		Nun wird der Widerstand solange verändert, bis keine Reflexion zu erkennen ist. Dieser Widerstand $R_v$ entspricht dann der Impedanz des Kabels. 
		Der Widerstand wird mittels 
		Voltmeter gemessen.\\
		Bei der Messung ergab das ein Widerstand von $R_v = Z = (50,0 \pm 1,0) \, \Omega$. 
		Der Fehler des Voltmeters wurde dem Datenblatt (\href{https://www.physik.uni-wuerzburg.de/fileadmin/11000000/03_Studium/02_Bachelor/Grundpraktikum/_imported/fileadmin/11016800/Geraete/VOLTCRAFT_VC_150_DMM_de_en_fr_nl.pdf}{Multimeter V150}) entnommen.
		Das ist mit dem Literaturwert von $Z = 50 \, \Omega$ vereinbar.


		\begin{figure}[H]
			\centering
			\subfloat[\centering max. Reflexion, $R_v = 2{,}8 \, \Omega$]{
				\includegraphics[angle=90, width=0.4\linewidth]{../Images/ALL0013/F0013TEK.JPG}
			}%
			\qquad
			\subfloat[\centering keine Reflexion, $R_v = 50 \, \Omega$]{
				\includegraphics[angle=90, width=0.4\linewidth]{../Images/ALL0014/F0014TEK.JPG}
			}%
			\qquad
			\subfloat[\centering wenig Reflexion, $R_v = 94{,}3 \, \Omega$]{
				\includegraphics[angle=90, width=0.4\linewidth]{../Images/ALL0015/F0015TEK.JPG}
			}%
			\caption{Messung mit Kohleschichtpotentiometer}
			\label{AnpassungKohle}
		\end{figure}

		\noindent Des Weiteren wurde die Anpassung für ein Wendelpotentiometer und eine Widerstandsdekade durchgeführt. Dabei ließ sich der reflektierte Impuls allerdings nicht 
		vollständig unterdrücken wie in Abb. \ref{AnpassungWendel} und Abb. \ref{AnpassungDekade} zu erkennen ist.

		\begin{figure}[H]
    		\centering
   			 \subfloat[\centering  $R_v = 0,1 \, \Omega$]{{\includegraphics[angle=90, width=0.4\linewidth]{../Images/ALL0016/F0016TEK.JPG} }}%
    		\qquad
   			 \subfloat[\centering $R_v = 50 \, \Omega$]{{\includegraphics[angle=90, width=0.4\linewidth]{../Images/ALL0017/F0017TEK.JPG} }}%
			\qquad
   			 \subfloat[\centering $R_v = 100,1 \, \Omega$]{{\includegraphics[angle=90, width=0.4\linewidth]{../Images/ALL0018/F0018TEK.JPG} }}%
   			 \caption{Messung mit Wendelpotentiometer}
   			 \label{AnpassungWendel}
		\end{figure}

		\begin{figure}[H]
    		\centering
   			 \subfloat[\centering $R_v = 0 \, \Omega$]{{\includegraphics[angle=90, width=0.4\linewidth]{../Images/ALL0019/F0019TEK.JPG} }}%
    		\qquad
   			 \subfloat[\centering $R_v = 50 \, \Omega$]{{\includegraphics[angle=90, width=0.4\linewidth]{../Images/ALL0020/F0020TEK.JPG} }}%
			\qquad
   			 \subfloat[\centering $R_v = 100 \, \Omega$]{{\includegraphics[angle=90, width=0.4\linewidth]{../Images/ALL0021/F0021TEK.JPG} }}%
   			 \caption{Messung mit Widerstandsdekade}
   			 \label{AnpassungDekade}
		\end{figure}

		\noindent Die Tatsache, dass die Reflexion nicht vollständig zu unterdrücken ist, ist dem Aufbau des Wendelpotentiometer geschuldet. Dieses beinhaltet nämlich ein
		spiralförmig gewickelten Widerstandsdraht welcher bei Stromfluss, wie eine Spule, ein ein Magnetfeld aufbaut und eine Spannung entgegen der Stromrichtung
		induziert. Das Wendelpotentiometer hat dadurch eine zusätzliche Impedanz.\\
		\\ 
		Nun wird ein $50 \, \Omega$-Stecker an das Ende des Koaxialkabels angeschlossen. Wie in Abb. \ref{Anpassung50} zu sehen, ist keine Reflexion erkennbar. Das
		liegt wieder daran, dass der Widerstand $R_v = Z$ der Impedanz entspricht (vgl. \ref{Reflexion}). \\
		Wird anstelle des $50 \, \Omega$-Steckers ein Kurzschlussstecker angeschlossen, so wird, wie in Abb. \ref{AnpassungKurz} zu sehen, das das reflektierte Signal invertiert.

		\begin{figure}[H]
			\centering
			\includegraphics[angle=90, width=0.7\linewidth]{../Images/ALL0023/F0023TEK.JPG}
			\caption{Anpassung mit $50\Omega$-Stecker}
			\label{Anpassung50}

		\end{figure}

		\begin{figure}[H]
			\centering
			\includegraphics[angle=90, width=0.7\linewidth]{../Images/ALL0022/F0022TEK.JPG}
			\caption{Anpassung mit Kurzschluss-Stecker}
			\label{AnpassungKurz}
		\end{figure}

		\subsubsection{Kombination von Abschlusswiderständen am Anfang und Ende des Kabels}

		Das Signal vom generatorferne Ende des Kabels wird nun auf CH I und das generatornahen Ende auf CH II gegeben. Des weiteren werden an den Kabelenden
		Widerstände $R_a$ (generatornah) und $R_e$ (generatorfern) angeschlossen. Für $R = 0$ wird ein Kurzschlussstecker, für $R = Z$ ein $50 \, \Omega-Stecker$ und 
		für $R = \infty$ ein offenes Kabelende verwendet. Es werden alle möglichen Kombinationen gemessen.

		\begin{table}[H]          %so funktionieren die Tabellen in LaTeX
		\centering
		\begin{tabular*}{\linewidth}{@{\extracolsep{\fill}}c|c|c}

		Kombination  	&     $R_a \, [\Omega]$  			&     $R_e \, [\Omega]$ \\
		\hline
  		 $i$ 			&     $0$ 		&     $0$ \\

		 $ii$ 			&     $0$ 		&     $50$ \\

		 $iii$ 			&     $0$ 		&     $\infty$ \\

		 $iv$ 			&     $50$ 		&     $0$ \\

		 $v$ 			&     $50$ 		&     $50$ \\

		 $vi$ 			&     $50$	 	&     $\infty$ \\

		 $vii$ 			&     $\infty$	&     $0$ \\

		 $viii$ 		&     $\infty$	&     $50$ \\

		 $ix$ 			&     $\infty$ 	&     $\infty$ \\

		\end{tabular*}  
		\caption{Kombinationen von Abschlusswiderständen}  %siehe Graphik: Beschriftung
		\label{AbschlussKombi}                             %siehe Graphik: zum Zitieren
		\end{table}

		\noindent Wenn nun $R_a = 0 \, \Omega$ wird hier das Signal am Kabelanfang reflektiert und invertiert und die Wellen überlagern sich (vgl. Abb. \ref{Kombi0}). 
		Wenn $R_a = Z$ ergibt sich mit dem Innenwiderstands des Funktionsgenerator 
		(vgl. \href{https://www.physik.uni-wuerzburg.de/fileadmin/11000000/03_Studium/02_Bachelor/Grundpraktikum/_imported/fileadmin/11016800/Geraete/Agilent-33220.pdf}{Anleitung Funktionsgenerator}) ein effektiver Widerstand von $R_{a,eff} = 25 \, \Omega$
		wodurch eine Reflexion entsteht. Die Amplitude wird reduziert (vgl. Abb. \ref{Kombi50}).
		Zu letzt wenn $R_a = \infty \, \Omega$ ergibt sich mit dem Funktionsgeneratorinnenwiderstand ein effektiver Widerstand $R_{a,eff} = 50 \, \Omega$ was der Impedanz entspricht,
		es gibt also keine Reflexion am Kabelanfang(vgl. Abb. \ref{KombiInf}).\\
		Zudem kommt aufgrund der Länge des Kabels noch eine Dämpfung der Amplitude hinzu, welche im folgenden Abschnitt genauer diskutiert wird.

		\begin{figure}[h]
    		\centering
   			 \subfloat[\centering i \label{i}]{{\includegraphics[angle=90, width=0.4\linewidth]{../Images/ALL0024/F0024TEK.JPG} }}%
    		\qquad
   			 \subfloat[\centering ii \label{ii}]{{\includegraphics[angle=90, width=0.4\linewidth]{../Images/ALL0025/F0025TEK.JPG} }}%
			\qquad
   			 \subfloat[\centering iii \label{iii}]{{\includegraphics[angle=90, width=0.4\linewidth]{../Images/ALL0026/F0026TEK.JPG} }}%
   			 \caption{Kombinationen mit $R_e = 0 \, \Omega$}
   			 \label{Kombi0}
		\end{figure}

		\begin{figure}[h]
    		\centering
   			 \subfloat[\centering iv \label{iv}]{{\includegraphics[angle=90, width=0.4\linewidth]{../Images/ALL0027/F0027TEK.JPG} }}%
    		\qquad
   			 \subfloat[\centering v \label{v}]{{\includegraphics[angle=90, width=0.4\linewidth]{../Images/ALL0028/F0028TEK.JPG} }}%
			\qquad
   			 \subfloat[\centering vi \label{vi}]{{\includegraphics[angle=90, width=0.4\linewidth]{../Images/ALL0029/F0029TEK.JPG} }}%
   			 \caption{Kombinationen mit $R_e = 50 \, \Omega$}
   			 \label{Kombi50}
		\end{figure}

		\begin{figure}[H]
    		\centering
   			 \subfloat[\centering vii \label{vii}]{{\includegraphics[angle=90, width=0.4\linewidth]{../Images/ALL0030/F0030TEK.JPG} }}%
    		\qquad
   			 \subfloat[\centering viii \label{viii}]{{\includegraphics[angle=90, width=0.4\linewidth]{../Images/ALL0031/F0031TEK.JPG} }}%
			\qquad
   			 \subfloat[\centering ix \label{ix}]{{\includegraphics[angle=90, width=0.4\linewidth]{../Images/ALL0032/F0032TEK.JPG} }}%
   			 \caption{Kombinationen mit $R_e = \infty \, \Omega$}
   			 \label{KombiInf}
		\end{figure}

		\subsubsection{Bestimmung der Ausbreitungsgeschwindigkeit durch Laufzeitmessung}

		Um die Ausbreitungsgeschwindigkeit in einem Koaxialkabel und in einem Verzögerungskabel zu bestimmen, wird die Laufzeit $\Delta t$ zwischen einlaufendem
		und reflektiertem Signal mittels der Cursor-Funktion des Oszilloskops gemessen. Die gemessenen Zeiten lassen sich Abb. \ref{ZeitKoax} und Abb. \ref{ZeitVerz}
		entnehmen. Der Fehler entsteht hierbei durch die Ablesegenauigkeit von $0,1 \, div \, \cdot \,$Zeitempfindlichkeit. Es ergibt sich: \\
		
		\begin{figure}[H]
			\centering
			\includegraphics[angle=90, width=0.5\linewidth]{../Images/ALL0033/F0033TEK.JPG}
			\caption{Messung der Laufzeit, Koaxialkabel}
			\label{ZeitKoax}
		\end{figure}

		\begin{equation*}
			\Delta t_{Koaxialkabel} = (520 \pm 50) \, ns 
		\end{equation*}

		\begin{figure}[H]
			\centering
			\includegraphics[angle=90, width=0.5\linewidth]{../Images/ALL0037/F0037TEK.JPG}
			\caption{Messung der Laufzeit, Verzögerungskabel}
			\label{ZeitVerz}
		\end{figure}

		\begin{equation*}
			\Delta t_{Verzögerungskabel} = (2,120 \pm 0,050) \, \mu s
		\end{equation*}

		\noindent Mithilfe der Länge L der Kabel lässt sich die Ausbreitungsgeschwindigkeit bestimmen:

		\begin{equation*}
			v = \frac{\Delta x}{\Delta t} = \frac{2 \cdot L}{\Delta t}
		\end{equation*}

		Es ergeben sich folgende Werte:

		\begin{equation*}
			v_{Koaxialkabel} = (192 \pm 18) \cdot 10^{6} \, \frac{m}{s}
		\end{equation*}
		\vspace{-2em}
		\begin{equation*}
			v_{Verzögerungskabel} = (459 \pm 12) \cdot 10^{3} \, \frac{m}{s}
		\end{equation*}

		\noindent Die Fehler ergeben sich mithilfe Gaußscherfehlerfortpflanzung. Da die Messung der Länge des Verzögerungskabels ungenau war, wurde eine Messreihe von
		fünf Messungen erstellt, daraus ergab sich auch entsprechender Längenfehler. Die Länge des Koaxialkabels wurde von der Kabelbeschriftung abgelesen und
		als fehlerfrei angenommen.

	\subsection{Messung der Ausbreitungsgeschwindigkeit mithilfe von stehenden Wellen}

		\subsubsection{Modellversuch zur Darstellung von stehenden Wellen}

		Um stehende Wellen zu erzeugen wird dem Oszilloskop das Triggersignal des Funktionsgenerators zugeführt. Die Buchsen A und B des Generators werden an die Eingänge
		CH I und CH II des Oszilloskops angeschlossen. An dem Generator kann man die Amplitude und die Geschwindigkeit des ausgehenden Signals einstellen. Die Mathematische
		Funktion des Oszilloskops wird genutzt, um die auf den Kanälen eingehenden Signale zu addieren. Damit wird die Überlagerung der Wellen simuliert.
		Wenn die Amplituden beider Wellen gleich sind, ist sowohl konstruktive (vgl. Abb \ref{321Konst}) als auch destruktive Interferenz (vgl. Abb \ref{321Dest})
		zu erkennen. Wenn die Wellen gegenläufig sind, gleiche Frequenz aber unterschiedliche Amplituden haben, ergibt sich eine stehende Welle
		mit asymmetrischer Amplitudenverteilung (vgl. Abb \ref{321Asym}).
		\begin{figure}[H]
			\centering
			\subfloat[\centering konst. Interferenz \label{321Konst}]{
				\includegraphics[width=0.4\linewidth]{../Images/ALL0034/F0034TEK.png}
			}%
			\qquad
			\subfloat[\centering asym. Amplitude \label{321Asym}]{
				\includegraphics[width=0.4\linewidth]{../Images/ALL0035/F0035TEK.png}
			}%
			\qquad
			\subfloat[\centering dest. Interferenz \label{321Dest}]{
				\includegraphics[width=0.4\linewidth]{../Images/ALL0036/F0036TEK.png}
			}%
			\caption{Modellversuch stehende Wellen}
			\label{AnpassungKohle}
		\end{figure}

		\subsubsection{Messung der Ausbreitungsgeschwindigkeit}
		\label{sec:Ausbreitungsgeschwindigkeit}

		Um die Ausbreitungsgeschwindigkeit einer Welle in deinem Koaxialkabel zu messen, wurd zunächst eine stehende, sinusförmige Welle erzeugt.
		Dazu wird wie schon aus den in den vorhergehenden Versuchen das Koaxialkabel mit dem einen Ende an den Funktionsgenerator angeschlossen,
		das andere Ende wieder mit einem T-Stecker an das Oszilloskop um dort Widerstände parallel zu schalten. Je nach Widerstand ergibt sich 
		durch Reflexion (vgl. \ref{Reflexion}) eine stehende Welle mit offenem Ende (kein Widerstasnd, kein Phasensprung) oder mit festem Ende 
		(Kurzschlussstecker, Phasensprung von $\pi$).\\
		Da das Koaxialkabel kein idealer Leiter ist, wird die Welle über den Verlauf im Kabel gedämpft. Dadurch ist die zu messende Amplitude an den 
		Knotenpunkten nicht mehr null, sondern nimmt ein endlichen Wert an. \\
		Nun wird die Frequenz so variiert, dass am Anfang (generatornah) des Kabels eine Knoten bzw. ein lokales Minimum ist. Das wird zwei mal bei
		offenem Ende getan (gemessen wird die Frequenz für $\frac{\lambda}{4}$, $\frac{3\cdot \lambda}{4}$) und einmal bei festem Ende 
		(Frequenz für $\frac{\lambda}{2}$). Die jeweils gemessenen Frequenzen lassen sich in Tab. \ref{TabAusbreitungsgeschwindigkeit} entnehmen. 
		Der Fehler der Frequenz wurde der Anleitung des Oszilloskops (\href{https://www.physik.uni-wuerzburg.de/fileadmin/11000000/03_Studium/02_Bachelor/Grundpraktikum/_imported/fileadmin/11016800/Geraete/Digitalspeicher-Oszilloskop-1001B.pdf}{Digitalspeicher-Oszilloskop 1001B}) entnommen.\\
		Die Wellenlängen ergeben sich hierbei mithilfe von Gl. \ref{WellenlängeOffen} und Gl. \ref{WellenlängeGeschlossen}. Die Länge des Koaxialkabels
		wurde von dessen Beschriftung abgelesen und als fehlerfrei angenommen.
		Mithilfe der Frequenzen und Wellenlänge wurde dann nach Gl. \ref{Ausbreitungsgeschwindigkeit} die Ausbreitungsgeschwindigkeit bestimmt, welche ebenfalls
		Tab. \ref{TabAusbreitungsgeschwindigkeit} zu entnehmen ist. \\
		Bei handelsüblichen Koaxialkabeln ist die Permeabilität $\mu_r = 1$ wodurch sich nach Gl. \ref{Ausbreitungsgeschwindigkeit2} die Dielektrizitätskonstante
		$\epsilon_r$ berechnen lässt.

		\begin{table}[H]          %so funktionieren die Tabellen in LaTeX
			\centering
			\begin{tabular*}{\linewidth}{@{\extracolsep{\fill}}c|c|c}

			$\lambda$ (in $m$) 					&     $f$ (in $10^{3} \, Hz$)	&     $v$ (in $10^6 \, \frac{m}{s}$)	\\
			\hline
			$\frac{\lambda}{4}$ 				&     ($938,0 \pm 1,9$)			&     $(187,60 \pm 0,38)$ 				\\

			$\frac{\lambda}{2}$ 				&     ($1892,0 \pm 5,2$)	 	&     $(189,20 \pm 0,52)$ 				\\

			$\frac{3 \cdot \lambda}{4}$ 		&     ($2875,0 \pm 1,9$) 		&     $(191,66 \pm 0,13)$ 				\\

			\end{tabular*}  
			\caption{Ausbreitungsgeschwindigkeit im Koaxialkabel} 			   %siehe Graphik: Beschriftung
			\label{TabAusbreitungsgeschwindigkeit}                             %siehe Graphik: zum Zitieren
		\end{table}
		\noindent Für die Elektrizitätskonstante ergibt sich:
		\begin{table}[H]          %so funktionieren die Tabellen in LaTeX
			\centering
			\begin{tabular}{c|c}

			$\lambda$ (in $m$) & $\epsilon_r$ \\
			\hline
			$\frac{\lambda}{4}$ & $(2,554 \pm 0,010)$ \\

			$\frac{\lambda}{2}$ & $(2,511 \pm 0,014)$ \\

			$\frac{3 \cdot \lambda}{4}$ & $(2,4466 \pm  0,0033)$ \\
			\end{tabular}  
			\caption{Elektrizitätskonstante in Wellenlängenabhängigkeit} 			   %siehe Graphik: Beschriftung
			\label{TabElektrizitaetskonstante}                             %siehe Graphik: zum Zitieren
		\end{table}

		\noindent Vergleicht man nun die gemessene Ausbreitungsgeschwindigkeit mit den Literaturwerten des RG214/U Kabels, zeigt sich, dass dieses Koaxialkabel
		eine Ausbreitungsgeschwindigkeit von $~66\%$\\ der Lichtgeschwindigkeit besitzt\cite{Koaxialkabel1}, während nach unsere Messungen es ~63\%-64\% der Lichtgeschwindigkeit
		sind. Die Begründung dieser Abweichung liegt in größtenteils an der Ungenauigkeit der Bestimmung der Resonanzfrequenz. Diese Unsicherheit rührt daher, 
		dass der Knotenpunkt durch Dämpfung nicht ein Nullpunkt, sondern eher ein kontinuierliches lokales Minimunm ist. Dadurch lässt sich die genaue Resonanzfrequenz 
		nicht gut einstellen. \\
		Das im Koaxialkabel verbaute Dielektrikum ist ein Polyethylen niedriger Dichte. Dieser Werkstoff besitzt eine 
		Dielektrizitätskonstante von $\epsilon_r = 2,4$ \cite{Koaxialkabel2}.
		Die gemessenen Werte sind in der gleichen Größenordnung, weichen allerdings auch innerhalb des Fehlers leicht von dem Literaturwert ab.
		Das kann daran liegen, dass das Datenblatt \cite{Koaxialkabel2} nicht auf die Frequenzabhängigkeit eingeht, in der Durchführung allerdings
		mit Frequenzen gemessen wurde, welche sich stark unterscheiden.

		\subsubsection{Bestimmung der Dämpfung des Signals im Koaxialkabel}
			Wie bereits in Abschnitt~\ref{sec:Ausbreitungsgeschwindigkeit} angedeutet ist der ohmsche Widerstand des Koaxialkabels nicht vernachlässigbar,
			da dieser eine Dämpfung auf das Eingangssignal ausübt.
			Diese Dämpfung ist Frequenzabhängig und soll nun quantitativ untersucht werden.
			Um den Zusammenhang zu verstehen wird zunächst das Dämpfungsmaß $D$ bestimmt. Es ist allgemein definiert über:
			\begin{equation*}
				D = 20 \cdot \log_{10} \frac{U(0)}{U(x)} \, dB
			\end{equation*}
			Die Spannungen als $U_{pp}$ (peak to peak) werden mit dem Oszilloskop bestimmt und das Dämpfungsmaß wird für verschiedene Wellenlängenverhältnisse berechnet.
			Die aus den Wellenlängenverhältnissen berechneten Frequenzen lassen sich aus Tabelle~\ref{TabAusbreitungsgeschwindigkeit} entnehmen.\\
			Die gedämpfte Spannung und das daraus berechnete Dämpfungsmaß sind (bei $U(0) = (160 \pm 25)\cdot 10^{-3} \, V$):
			\begin{table}[htbp]          %so funktionieren die Tabellen in LaTeX
				\centering
				\begin{tabular*}{\linewidth}{@{\extracolsep{\fill}}c|c|c}
				$\lambda$ (in $m$) 					& $U(x)$ (in $10^{-3} \, V$) & $D$ (in $dB$)\\
				\hline
				$\frac{\lambda}{4}$ 				& $(14,20 \pm 0,45)$ & $(21,0 \pm 1,4)$\\
				$\frac{\lambda}{2}$ 				& $(21,8 \pm 1,1)$ & $(17,3 \pm 1,4)$\\
				$\frac{3 \cdot \lambda}{4}$ 		& $(26,4 \pm 1,1)$ & $(15,7 \pm 1,4)$\\
				\end{tabular*}  
				\caption{Dämpfung im Koaxialkabel} 			   %siehe Graphik: Beschriftung
				\label{TabU-D}                             %siehe Graphik: zum Zitieren
			\end{table}
			\\
			\begin{figure}[H]
				\centering
				\includegraphics[width=0.8\linewidth]{../Images/DoverF.png}
				\caption{Dämpfung in Abh. der Frequenz}
				\label{Img:D-F}
			\end{figure}
			\noindent In Abbildung~\ref{Img:D-F} ist deutlich zu erkennen, dass das Dämpfungsmaß von der Frequenz abhängt.

	\subsection{Messung der Lichtgeschwindigkeit}

	Die Lichtgeschwindigkeit wird mittels Laufzeitmessung bestimmt. Der Versuchsaufbau ist Abb. \ref{AufbauLichtgeschwindigkeit}  zu entnehmen.
<<<<<<< HEAD
	Eine Hochleistungs-LED sendet sehr kurze, rote Lichtpulse von einer Breite von
	$20 \, ns$ aus. Diese Signale werden nach hin- und zurücklaufen der Messstrecke $s$ in
=======
	\begin{figure}[H]
	\centering
		\includegraphics[width=0.7\linewidth]{../Images/Screenshot from 2025-11-20 16-54-51.png}
	\caption{Aufbau für die Messung der Lichtgeschwindigkeit}
	\label{AufbauLichtgeschwindigkeit}
	\end{figure}
	Eine Hochleistungs-LED sendet sehr kurze, rote Lichpulse von einer Breite von
	20ns aus. Diese Signale werden nach hin- und zurücklaufen der Messstrecke s in
>>>>>>> a2f921f55d758d863c1fe933570d0a069be4a827
	Spannungspulse umgewandelt, welche am Oszilloskop beobachtet werden.
	Das Fenser $F_1$ der Linse L bildet die Lichtquelle in das Unendliche ab. Der 
	Tripelspiegel $T_1$ faltet den Strahlengang in sich zurück, sodass die LED auf
	sich selbst abgebildet wird. Der Strahlteiler $S$ reflektiert das von $T_1$ 
	kommende Lichtsignal auf die Photodiode $D$. Zudem reflektiert dieser die Hälfte 
	des Lichts nach oben durch das Fenster $F_2$ auf den Tripelspiegel $T_2$ welcher
	das Signal ebenfalls auf die Photodiode $D$ sendet, was als Referenzsignal dient.
	Diese Laufzeitverzögerung ist allerdings vernachlässigbar klein. \\
	Die Laufzeit wird für 9 verschiedene Lichtwege gemessen. Dabei ist darauf zu achten,
	dass die LED bei der Startposition $s_{LED} = (23,150 \pm 0,050) \, cm$ steht.
	Daraus lässt sich die Lichtgeschwindigkeit berechnen:
	\begin{equation*}
		c = \frac{s}{\Delta t}
	\end{equation*}
	Der Fehler der Lichtgeschwindigkeit ergibt sich durch Gaußschefehlerfortpflanzung.
	Der letztendliche Wert der Lichtgeschwindigkeit ergibt sich aus dem Bestwert der
	Messreihe: 
	\begin{equation*}
		c = (293 \pm 16) \cdot 10^{6} \, \frac{m}{s}
	\end{equation*}

	Der Bestwert stimmt mit dem Literaturwert $c_{Literaturwert} = 299792458 \, \frac{m}{s}$ \cite{Lichtgeschwindigkeit} (also $c \approx 300 \cdot 10^{6} \, \frac{m}{s}$)
	innerhalb des Fehlers überein. Bei manchen der Messungen wurde allerdings Überlichtgeschwindigkeit gemessen,
	was unphysikalisch ist. Das liegt an der Ungenauigkeit in der Messmethode.

\section{Zusammenfassung}

Der Wellenwiderstand eines Koaxialkabels wurde durch Reflexionsunterdrückung
mittels Kohleschichtpotentiometer am Kabelende bestimmt. Dieser beläuft sich auf
$Z = (50 \pm 1)\Omega$. Diese Messung wurde des weiteren mit einem Wendelpotentiometer als auch mit einer Widerstandsdekade durchgeführt. Hier war zu beobachten,
dass es trotz einstellen der Impedanz, zu Reflexion kam. Bei Verwendung des Kurzschlusssteckers wurde eine Reflexion mit invertierter Amplitude festgestellt.\\
Des weiteren wurde die Ausbreitungsgeschwindigkeit in dem Koaxialkabel mittels Laufzeitmessung bestimmt. Das ergab:
\begin{equation*}
	v_{koaxial} = (192 \pm 18) \cdot 10^6 \, \frac{m}{s}
\end{equation*}
 Die Ausbreitungsgeschwindigkeit wurde zudem auch noch mithilfe von stehenden Wellen für drei unterschiedliche Wellenlängen bestimmt:
 \begin{equation*}
	v_{\frac{\lambda}{4}} = (187,59820 \pm 0,00063) \cdot 10^6 \, \frac{m}{s}
\end{equation*}
\begin{equation*}
	v_{\frac{\lambda}{2}} = (189,19800 \pm 0,00044) \cdot 10^6 \, \frac{m}{s}
\end{equation*}
\begin{equation*}
	v_{\frac{3\lambda}{4}} = (191,66467 \pm 0,00036) \cdot 10^6 \, \frac{m}{s} 
 \end{equation*}
Bei der Messung der mittels stehender Welle ist spannend zu beobachten, dass hier der Effekt der Dispersion merklich zum tragen kommt.
Durch geringere Fehlerquellen ist die Messung mit stehenden Wellen auch genauer als die über die Laufzeitmessung. \\
Zudem wurde die Ausbreitungsgeschwindigkeit in einem Verzögerungskabel mithilfe der Laufzeitmessung gemessen:
\begin{equation*}
	v_{verzögerungskabel} = (459 \pm 12) \cdot 10^3 \, \frac{m}{s}
\end{equation*}
Ebenfalls durch Laufzeitmessung für verscheidene Distanzen wurde die Lichtgeschwindigkeit gemessen. Es ergibt sich ein Wert von
\begin{equation*}
	c = (293 \pm 16)\cdot 10^6 \, \frac{m}{s}
\end{equation*}
welcher mit dem Literaturwert überein stimmt.
\begin{thebibliography}{}    %so wird das Literaturverzeicnis erstellt
\bibitem{Lichtgeschwindigkeit}
	\url{https://physics.nist.gov/cgi-bin/cuu/Value?c}, besucht am 18.11.2025 um 16.29 Uhr
\bibitem{Koaxialkabel1}
	\url{https://docs.rs-online.com/75d1/A700000011339821.pdf}, besucht am 20.11.2025 um 00.59 Uhr
\bibitem{Koaxialkabel2}
	\url{https://www.auer-kunststofftechnik.de/pdf/Technisch.Datenblatt\%20PE\%20-LD\%20natur\%2070093.pdf}, besucht am 20.11.2025 um 08:05 Uhr
\end{thebibliography}

\end{document}